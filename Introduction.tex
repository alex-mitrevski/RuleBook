%% %%%%%%%%%%%%%%%%%%%%%%%%%%%%%%%%%%%%%%%%%%%%%%%%%%%%%%%%%%%%%%%%%%%%%%%%%%%
%%
%%    author(s): RoboCupAtHome Technical Committee(s)
%%  description: Introduction
%%
%% %%%%%%%%%%%%%%%%%%%%%%%%%%%%%%%%%%%%%%%%%%%%%%%%%%%%%%%%%%%%%%%%%%%%%%%%%%%
\chapter{Introduction}
\label{chap:introduction}


\section{RoboCup}
\label{sec:introduction:robocup}
\RoboCup{} is an international joint project to promote AI, robotics, and related fields. It is an attempt to foster AI and intelligent-robotics research by providing standard problems where a wide range of technologies can be integrated and examined. More information can be found at:
{\small\url{http://www.robocup.org/}}.


\section{RoboCup@Home}
\label{sec:introduction:robocupathome}
The \AtHome{} league aims to develop service and assistive robot technology with high relevance for future personal domestic applications. It is the largest international annual competition for autonomous service robots and is part of the \RoboCup{} initiative. A set of benchmark tests is used to evaluate the robots' abilities and performance in a realistic non-standardized home environment setting. Focus lies on, but is not limited to, the following domains: \HRI{} and Cooperation, \NAV{} and \MAP{} in dynamic environments, \CV{} and \OR{} under natural lighting conditions, Object \MAN{}, \AB{}, \BI{}, \AmI{}, Standardization and \SysI{}. It is co-located with the \RoboCup{} symposium.

%%%%%%%%%%%%%%%%%%%%%%%%%%%%%%%%%%%%%%%%%%%%%%%%%%%%%%%%%
\section{Competition Procedure}
\label{sec:rules:competitionprocedure}

A \RoboCup\AtHome{} competition consists of the following stages:

\begin{enumerate}
	\item \textbf{\RobotInspection{}:} For security, robots are inspected during \SetupDays{}.
	All registered teams can participate.

	\item \textbf{\SONE{}:} First set of tests, assessing the robot's basic abilities.
	Only teams that passed the \RobotInspection{} can participate.

	\item \textbf{\STWO{}:} Second set of tests, assessing more complex abilities and behaviors.
	The best \SI{50}{\percent} of teams (after \SONE{}) can participate. If the total number of teams is less than 12, up to 6 teams may advance to \STWO{}.

	\item \textbf{\FINAL{}:} An open demonstration, asking teams to showcase complex behaviors and novel approaches. The two best scoring teams (\SONE{} and \STWO{} combined) can participate.
\end{enumerate}

\begin{table}[h]
	\newcolumntype{C}[1]{>{\centering\let\newline\\\arraybackslash\hspace{0pt}}m{#1}}
	\newcolumntype{S}{C{1.6cm}}
	\newcolumntype{M}{C{3.2cm}}
	\begin{center}
		\begin{tabularx}{14.56cm}{S|S|S|S|S|S|S|S}
			\hline
			\multicolumn{2}{|M|}{ \cellcolor[HTML]{FFFFC7}Setup Days} &
			\multicolumn{2}{M|}{ \cellcolor[HTML]{67FD9A}\iterm{Stage~I}} &
			\multicolumn{2}{M|}{ \cellcolor[HTML]{9698ED}\iterm{Stage~II}} &
			\multicolumn{2}{M|}{ \cellcolor[HTML]{FFCCC9}\iterm{Finals}}\\
			\hline
			%Second row
			\multicolumn{1}{S|}{} &
			\multicolumn{2}{M|}{$\xrightarrow{advance}$\newline All teams that \newline passed Inspection} &
			\multicolumn{2}{M|}{$\xrightarrow{advance}$\newline Best 6 ($<12$) \newline or best 50\% ($\geq 12$)} &
			\multicolumn{2}{M|}{$\xrightarrow{advance}$\newline Best 2 \newline teams} &
			\multicolumn{1}{C{1.2cm}}{~}
			\\ \cline{2-7}
		\end{tabularx}
	\end{center}
\end{table}

\noindent In case of having no considerable score difference between a team advancing to the next stage and a team not advancing, the \TC{} may announce additional teams that progress to the next stage.

\subsection{Scenarios}
\label{sec:rules:scenarios}
The tests in \SONE{} and \STWO{} are divided in two thematic scenarios:
\begin{itemize}
	\item \textbf{\Housekeeper{}:} Features tests related to cleaning, organizing, and maintaining.
	
	\item \textbf{\Partyhost{}:} Focuses on providing general assistance during a party by attending the needs of the guests.
\end{itemize}


\subsection{Schedule}
\label{sec:rules:schedule}
There are two \Testblocks{} in a competition day. Each block has a stage, and one or two thematic scenarios assigned. An exception is the \emph{Restaurant} test (see~\ref{test:restaurant}) which has its own block. During a block, each team has at least two \Testslots{} available, where they can choose which test, fitting the stage and scenario, they want to perform. The teams must inform the \OC{} of which tests they will perform a day prior, usually in the \TLM{} (see~\ref{sec:rules:teamleadermeeting}). Teams have to indicate to the \abb{OC} when they are skipping a \Testslot{}. Without such indication, they may receive a penalty (see~\ref{sec:rules:missingslot}).

% Please add the following required packages to your document preamble:
% \usepackage[table,xcdraw]{xcolor}
% If you use beamer only pass "xcolor=table" option, i.e. \documentclass[xcolor=table]{beamer}
\begin{table}[H]
	\centering\small
	\newcommand{\teams}[2]{%
		\tiny
		\begin{tabular}{c}%
			\textit{Slot $1$, Team $#1$}\\
			$\vdots$\\
			\textit{Slot $N$, Team $#2$}\\
			\textit{Slot $N+1$, Team $#1$}\\
			$\vdots$\\
			\textit{Slot $2N$, Team $#2$}\\
		\end{tabular}
	}
	\newcommand{\wcell}[2]{%
		\parbox[c]{2.5cm}{%
			\vspace{#1}%
			\centering%
			#2%
			\vspace{#1}%
		}%
	}
	\newcommand{\cell}[1]{\wcell{0.2\baselineskip}{#1}}
	% \newcommand{\mr}[1]{\multirow{2}{*}{#1}}


	\begin{tabular}{
		>{\centering\arraybackslash}m{2.5cm}|%
		>{\columncolor[HTML]{9AFF99}}c |%
		>{\columncolor[HTML]{9AFF99}}c |%
		>{\columncolor[HTML]{CBCEFB}}c |%
		>{\columncolor[HTML]{FF8D27}}c  %
	}
	\multicolumn{1}{ c }{}
		& \multicolumn{1}{ c }{\cellcolor{white} Day 1 }
		& \multicolumn{1}{ c }{\cellcolor{white} Day 2 }
		& \multicolumn{1}{ c }{\cellcolor{white} Day 3 }
		& \multicolumn{1}{ c }{\cellcolor{white} Day 4 }
		\\\hhline{~---~}

	\cell{Block 1\\\footnotesize(9:00--12:00)}
		& \cell{Housekeeper\\\teams{i}{j}}
		& \cell{Housekeeper\\~\\Party Host}
		& \cell{Restaurant}
		& \cellcolor{white}
		\\\hhline{~----}



	\multicolumn{1}{ c }{}
		& \multicolumn{3}{ c }{\wcell{0.5\baselineskip}{\color{gray}Lunch}}
		& \multicolumn{1}{|c|}{\cellcolor[HTML]{FF8D27}\cell{\textbf{Finals}}}
		\\\hhline{~----}

	\cell{Block 2\\\footnotesize(14:00--17:00)}
		& \cell{Party Host\\\teams{k}{l}}
		& \cellcolor[HTML]{CBCEFB}\cell{Party Host}
		& \cell{Housekeeper}
		& \cellcolor{white}
		\\\hhline{~---~}

	\multicolumn{1}{ c }{}
		& \multicolumn{1}{ c }{\wcell{0.5\baselineskip}{\color[HTML]{029734}Stage 1}}
		& \multicolumn{1}{ c }{\cellcolor{white}}
		& \multicolumn{1}{ c }{\wcell{0.5\baselineskip}{\color[HTML]{6668e5}Stage 2}}\\
	\end{tabular}

	\caption{Example schedule.
		Each of the $N$ teams has two slots assigned per block.
		At least two blocks are scheduled per day with assigned themes.
	}
	\label{tbl:schedule}
\end{table}

\noindent\textbf{Note:} The schedule will be announced during \SetupDays{} (see~\ref{chap:setupdays}) by the \abb{OC}.


\subsection{Team Leader Meeting}
\label{sec:rules:teamleadermeeting}
In the evening before each competition day, a \TLM{} is held. Attendance from all teams participating in the next day's tests is mandatory. During the meeting, teams can ask questions and discuss the upcoming tests with the \abb{TC} and \abb{OC}. The starting time will be announced by the \abb{OC}. Decisions made in the \abb{TLM} are binding. The \abb{TC} and referees on site will decide on anything coming up during or after a test.

\subsection{Scoring System}
\label{sec:rules:scoringsystem}
Each test has a main objective and a set of bonuses.
To score in a test, a team must accomplish the main goal (in parts if allowed). Bonuses are only given if at least \SI{50}{\percent} of the points for the main goal are achieved. Overall scoring in a stage is calculated as the sum of the maximum score obtained in each test. The final score is calculated differently and is normalized (see~\ref{sec:finals:scoring}). A team cannot get a negative score for a test unless a penalty was received.

\paragraph*{Note: } Once a scoresheet has been signed by the team leader or the scores have been published, the \abb{TC} decision is irrevocable.

% Local Variables:
% TeX-master: "../Rulebook"
% End:


\input{introduction/Infrastructure}

%% %%%%%%%%%%%%%%%%%%%%%%%%%%%%%%%%%%%%%%%%%%%%%%%%%%%%%%%%%%%%%%%%%%%%%%%%%%%
%%
%%    author(s): RoboCupAtHome Technical Committee(s)
%%  description: Introduction - Leagues
%%
%% %%%%%%%%%%%%%%%%%%%%%%%%%%%%%%%%%%%%%%%%%%%%%%%%%%%%%%%%%%%%%%%%%%%%%%%%%%%
\section{Leagues}
\label{sec:introduction:leagues}

\AtHome{} is divided in three leagues. Two are \SPLs{}, where each team uses the same robot platform, and one is the \OPL{}, where teams are free to choose their robot. The official leagues and their names are:
\begin{itemize}
  \item \DSPL{}
  \item \SSPL{}
  \item \OPL{}
\end{itemize}

\noindent All leagues share the same set of rules. The \DSPL{} uses the \HSR{} platform shown in figure \ref{fig:toyotaHSR} and the \SSPL{} uses the \PEPPER{} platform shown in figure \ref{fig:softbank-pepper}.

\begin{minipage}{0.5\textwidth}
	\begin{figure}[H]
		\begin{center}
			\includegraphics[height=0.6\textwidth]{images/toyota_hsr.png}
			\caption{Toyota HSR}
			\label{fig:toyotaHSR}
		\end{center}
	\end{figure}
\end{minipage}
\begin{minipage}{0.5\textwidth}
	\begin{figure}[H]
		\begin{center}
			\includegraphics[height=0.6\textwidth]{images/softbank_pepper.png}
			\caption{Softbank / Aldebaran Pepper}
			\label{fig:softbank-pepper}
		\end{center}
	\end{figure}
\end{minipage}


%% %%%%%%%%%%%%%%%%%%%%%%%%%%%%%%%%%%%%%%%%%%%%%%%%%%%%%%%%%%%%%%%%%%%%%%%%%%%
%%
%%    author(s): RoboCupAtHome Technical Committee(s)
%%  description: Introduction - Competition
%%
%% %%%%%%%%%%%%%%%%%%%%%%%%%%%%%%%%%%%%%%%%%%%%%%%%%%%%%%%%%%%%%%%%%%%%%%%%%%%
\section{Competition}
The competition consists of two \emph{Stages} and the \FINAL{}. Each stage comprises a series of \iterm{Tests}. The best teams from \SONE{} advance to \STWO{}, with more difficult tests. The competition ends with the \FINAL{} where the two highest ranked teams of each league compete to win.


%% %%%%%%%%%%%%%%%%%%%%%%%%%%%%%%%%%%%%%%%%%%%%%%%%%%%%%%%%%%%%%%%%%%%%%%%%%%%
%%
%%    author(s): RoboCupAtHome Technical Committee(s)
%%  description: Introduction - Awards
%%
%% %%%%%%%%%%%%%%%%%%%%%%%%%%%%%%%%%%%%%%%%%%%%%%%%%%%%%%%%%%%%%%%%%%%%%%%%%%%
\section{Awards}
\label{sec:introduction:awards}
All the awards need to be approved by the \RCF{}. Not all awards must be given.


\paragraph{Winner of the Competition}
\label{sec:introduction:winner}
Each league has 1st, 2nd, and 3rd place award trophies. If eight or fewer teams are participating, no 3rd place award trophy is given.

\newpage
\paragraph*{Note: } For the following awards, the \EC{} nominates a set of candidates from which the \TC{} elects the winner. One cannot nominate or vote for their own team.

\paragraph{Best Human-Robot Interface Award}
\label{sec:introduction:hriaward}
To honour outstanding Human-Robot Interfaces, the \HRIAward{} may be given to one of the participating teams. It is especially important that the interface is open and available to the \AtHome{} community.

\paragraph{Best Poster}
\label{sec:introduction:bestposter}
To foster scientific knowledge exchange and reward a teams' effort to present their contributions, all scientific posters of each league are eligible to receive the \DSPLPosterAward, \SSPLPosterAward, or \OPLPosterAward, respectively.

Posters are graded on presenting innovative and state-of-the-art research within a field with direct application to \RoboCup\AtHome{} in an appealing, easy-to-read way, while demonstrating successful and clear results. In addition to be attractive and well-rated in the \PS{} (see~\ref{sec:setupdays:postersession}), the explained research must have impact in the team's performance during the competition.

\paragraph*{Note: } For the following award, the \TC{} and team leaders nominate a set of candidates from which the \EC{} elects the winner. One cannot nominate or vote for their own team.

\paragraph{Open-Source Software Award}
\label{sec:introduction:assaward}
For promoting software exchange and collaboration, \AtHome{} awards the best open source software contributions to the community. The software must be easy to read, properly documented, follow standard design patterns, be actively maintained, and meet IEEE software engineering metrics of scalability, portability, maintainability, fault tolerance, and robustness. In addition, the open sourced software must be made available as a framework-independent standalone library so it can be reused with any software architecture.

Candidates must send their application to the \TC{} at least one month before the competition in form of a short paper (max 4 pages), following the same format used for the \TDP{} (see~\refsec{sec:rules:application:tdp}). The paper should include a brief explanation of the approach, comparison with State-of-the-Art techniques, statement of the used metrics and software design patterns, and the name of the teams and other collaborators that are also using the software.


\paragraph{Open Challenge Award}
\label{sec:introduction:ocaward}

To encourage teams to present their research to the rest of the league, \AtHome{} grants the \OCAward{} to the best open demonstration presented during the competition. This award is granted only if a team has demonstrated innovative research that is related to the global objectives of \AtHome{}. 


\paragraph{Skill Certificates}
\label{award:skill}
The @Home league features certificates for the robots best at one of the skills below:
\begin{itemize}
	\item Navigation
	\item Manipulation
	\item Speech Recognition
	\item Person Recognition
\end{itemize}

A team is given the certificate if it scored at least 75\% of the attainable points for that skill.
This is counted over all tests and challenges, so e.g.~if the robot scores manipulation points during the Storing Groceries test, that will count towards the Best in Manipulation certificate.
The certificate will only be handed out if the team is \emph{not} the overall winner of the competition.


% Local Variables:
% TeX-master: "Rulebook"
% End:
