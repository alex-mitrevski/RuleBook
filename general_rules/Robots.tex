%%%%%%%%%%%%%%%%%%%%%%%%%%%%%%%%%%%%%%%%%%%%%%%%%%%%%%%%%
\section{Robots}
\label{sec:rules:robots}

\subsection{Number of Robots}
\label{sec:rules:robotsnumber}

\begin{itemize}
	\item \textbf{Registration:} The maximum number of robots used in tests per team is \emph{two} (2).
	\item \textbf{Regular Tests:} Only one robot is allowed per test run. For different test runs, different robots can be used.
	\item \textbf{Open Demonstrations:} In the \iterm{Finals} both robots can be used simultaneously.
\end{itemize}

\subsection{Appearance and Safety}
\label{sec:rules:robotappearance}

Robots should have a product-like appearance and be safe to operate. The following rules apply to all robots:
\begin{itemize}
	\item \textbf{Cover:} The robot's internal hardware (electronics and cables) should be covered so that safety is ensured. The use of (visible) duct tape is strictly prohibited.
	\item \textbf{Loose cables:} Loose cables hanging out of the robot are not permitted.
	\item \textbf{Safety:} The robot must not have sharp edges or elements that might harm people.
	\item \textbf{Annoyance:} The robot must not be continuously making loud noises or use blinding lights.
	\item \textbf{Marks:} The robot may not exhibit any kind of artificial marks or patterns.
	\item \textbf{Driving:} To be safe, the robots should be careful when driving. Obstacle avoidance is mandatory.
\end{itemize}

The compliance with these rules will be verified during \RobotInspection{} (see \ref{sec:setupdays:inspection}).

\subsection{Standard Platform Leagues}
\label{sec:rules:robotappearance_spl}
For Robots competing in a \SPL{}, modifications and alterations to the robots are strictly forbidden. This includes, but is not limited to, attaching, connecting, plugging, gluing, and taping components into and onto the robot, as well as, modifying or altering the robot structure. Not complying with this rule leads to an immediate disqualification and penalization of the team (see~\ref{sec:rules:penaltiesbonuses}).

Robots are allowed to \enquote{wear} clothes, have stickers (e.g., a sticker exhibiting the logo of a sponsor), and be painted as long as they are compliant with section \ref{sec:rules:robotappearance}.

\subsubsection{DSPL Modifications}
\label{sec:rules:mountingbracket}
In the \DSPL{}, some modifications to the \HSR{} are allowed. An official \MountingBracket{} is provided by Toyota for the \HSR{}. Any laptop fitting inside the \MountingBracket{} may be used as additional on-board computing. Furthermore, teams are allowed to attach the following devices to the robot or the laptop in the \MountingBracket{}:
\begin{enumerate}
	\item \textbf{Audio:} USB audio output device, e.g. USB-powered speaker, possibly with sound card.
	\item \textbf{Wi-Fi Adapter:} USB-powered IEEE 802.11ac (or newer) compliant device.
	\item \textbf{Ethernet Switch:} USB-powered IEEE 802.3ab (or newer) compliant device.
\end{enumerate}

\noindent A maximum of three such devices may be attached, they must not increase the robot's dimension.


\input{general_rules/Robots-OPL}



% Local Variables:
% TeX-master: "../Rulebook"
% End:
